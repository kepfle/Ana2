\chapter{Differentialgleichungen}
\changesection
\section{Definitionen und Beispiele}
\section{Lineare Differentialgleichungen erster Ordnung}
\begin{beispiel*}[Logistische Differentialgleichung]
	\[ P'=\lambda KP-\lambda P^2 \]
	$ u\coloneqq-\frac{1}{P} $ l\"ost $ u'=\frac{P'}{P^2}=-\lambda Ku-\lambda $.
	\[ u(t)=u(0)e^{-\lambda Kt}-\frac{1}{k}\left(1-e^{-\lambda Kt}\right) \]
	\[ P(t)=-\frac{1}{u(t)}=\frac{KP(0)}{(K-P(0))e^{-\lambda Kt}+P(0)}\leq K\text{ wenn }P(0)\leq K \]
	F\"ur $ P(0)>K $, ist 
	\[ P(t)=\frac{KP(0)}{Ke^{-\lambda Kt}+(1-e^{-\lambda Kt})P(0)}\leq P(0). \]
\end{beispiel*}
\section{L\"osungsmethoden f\"ur spezielle Differentialgleichungen erster Ordnung}
\changesubsection
\subsection{Bernoulli-Differentialgleichung}
Seien $ a,b\in C^0(I) $, $ \rho\in\R\setminus\lbrace 0,1\rbrace $. Die DGL
\[ (B)\qquad u'=au+bu^\rho \]
hei\ss t \deftxt{Die Bernoulli-Differentialgleichung}.\\
Im Allgemeinen ist $ (B) $ nur f\"ur positive Funktionen $ u $ erkl\"art.
\begin{beispiel*}
	Die Logistische DGL: $ \rho=2 $.
\end{beispiel*}
\begin{lemma}
	\bullshit
	\begin{enumerate}
		\item Ist $ z\in C^1(I) $ positiv und l\"ost \[ (L)\qquad z'=(1-\rho)(az+b) \], so ist $ u=\frac{1}{z^{1-\rho}} $ L\"osung zu $ (B) $.
		\item Ist $ u\in C^1(I) $ positiv und l\"ost $ (B) $, so ist $ z=u^{1-\rho} $ L\"osung zu $ (L) $.
		\item Sind $ u_1,u_2 $ positive L\"osungen zu $ (B) $ und gibt es ein $ t_0\in I $ mit $ u_1(t_0)=u_2(t_0) $, so ist schon $ u_1=u_2 $.
 	\end{enumerate}
\end{lemma}
\begin{beweis}
	\begin{enumerate}
		\item Sei $ u=\frac{1}{z^{1-\rho}} $. Dann
		\[ u'=\frac{1}{1-\rho}z^{\frac{1}{1-\rho}-1}z'=z^{\frac{\rho}{1-\rho}}(az+b)=az^{\frac{\rho}{1-\rho}+1}+bz^{\frac{1}{1-\rho}\rho}=au+bu^\rho. \]
		\item 
		\[ z'=(u^{1-\rho})'=(1-\rho)u^{-\rho}u'=(1-\rho)u^{-\rho}(au+bu^{\rho})=(1-\rho)(az+b) \]
		\item $ z_k\coloneqq u_k^{1-\rho} $ l\"osen beide $ z_k'=(1-\rho)(az_k+b) $; $ w\coloneqq z_1-z_2 $ l\"ost $ w'=(1-\rho)aw $ und $ w(t_0)=0 $. Also $ w=0 $, $ z_1=z_2 $ und somit $ u_1=u_2 $.
	\end{enumerate}
\end{beweis}
\begin{lemma}
	Sei $ \rho\in\Z $, $ \rho\geq 2 $ und $ w $ eine beliebige L\"osung zu $ (B) $ (also nur $ w\in C^1(I) $), so gilt: Ist $ t_0\in I $, $ w(t_0)=0 $, so $ w=0 $.
\end{lemma}
\begin{beweis}
	Sei $ \delta>0 $, $I_\delta\coloneqq [t_0-\delta,t_0+\delta]\subset I $; $ M_\delta\coloneqq\max_{I_\delta} |w| $, $ L_\delta\coloneqq\max_{I_\delta}|a|+M_\delta^{\rho-1}\max_{I_\delta}|b| $. Wir zeigen: $ \forall n\geq 0 $ ist \[ |w(t)|\leq\frac{M_\delta L_\delta^n}{n!}|t-t_0|^n,\quad t\in I_\delta. \]
	\begin{description}
		\item[Induktion nach $ n $:] $ n=0\surd $;\\
		$ n-1\to n: $ \begin{align*}
		w(t)&=\int_{t_0}^t w'(s)\dd s\\
		&=\int_{t_0}^taw(s)-bw(s)^\rho\dd s\\
		&=\int_{t_0}^t w(s)(a(s)+b(s)w(s)^{\rho-1})\dd s
		\end{align*}
		\[ |w(t)|\leq L_\delta\int_{t_0}^t |w(s)|\dd s\leq L_\delta\frac{M_\delta L_\delta^{n-1}}{(n-1)!}\int_{t_0}^t (s-t_0)^{n-1}\dd s=\frac{M_\delta L_\delta^n}{n!}(t-t_0)^n \]
		f\"ur $ t\geq t_0 $. Analog f\"ur $ t\leq t_0 $.
	\end{description}
	Also $ w=0 $ auf $ I_\delta $. Genauso zeige: Ist $ t_\ast\in I $, $ w(t_\ast)=0 $, so $ \exists\delta_\ast>0 $: $ w=0 $ auf $ [t_\ast-\delta_\ast,t_\ast+\delta_\ast] $. Dann ist $ \lbrace t\in I\mid w(t)=0\rbrace $ offen und abgeschlossen und $ \neq\emptyset $, stimmt also mit $ I $ \"uberein.
\end{beweis}
\subsection{Die Riccati-Differentialgleichung}
Sind $ a,b,f\in C^0(I) $, so hei\ss t die DGL
\[ (R)\qquad u'=au+bu^2+f \]
\deftxt{Riccatische Differentialgleichung}.
\begin{lemma}
	Sei $ u_p\in C^1(I) $ eine L\"osung zu $ (R) $.
	\begin{enumerate}
		\item Dann ist eine Funktion $ u\in C^1(I) $ L\"osung zuz $ (R) $, wenn $ w=u-u_p $ L\"osung zu
		\[ (B^\ast)\qquad w'=(a+2bu_p)w+bw^2 \]
		ist.
		\item L\"osen $ u_1,u_2 $ die DGL $ (R) $ und $ u_1(t_0)=u_2(t_0) $ f\"ur ein $ t_0\in I $, so $ u_1=u_2 $.
	\end{enumerate}
\end{lemma}
\begin{beweis}
	\begin{enumerate}
		\item
	\[ w'=u'-u_p'=au+bu^2+f-au_p-bu_p-f=aw+bw(u+u_p)=aw+bw(w+2u_p)=(a+2bu_p)w+bw^2 \]
	Umgekehrt sei $ w $ L\"osung zu $ (B^\ast) $; $ u\coloneqq w+u_p $. Dann
	\[ u'=w'+u'_p=(a+2bu_p)w+bw^2+au_p+bu^2_p+f=au+b(w^2+2u_p w+u_p^2)+f=au+bu^2+f \]
	\item $ w=u_1-u_2 $ l\"ost $ (B^\ast) $ mit $ w(t_0)=0\Rightarrow w\equiv 0 $.
\end{enumerate}
\end{beweis}
\begin{beispiel*}[Grenzgeschwindigkeit eines Autos]
	Die Geschwindigkeit $ v $ eines Autos erf\"ullt $ v'=f-\rho v^2 $ (mit $ f,\rho>0 $). Die DGL ist vom Typ $ (R) $ mit $ a=0 $, $ b=-\rho $. $ v_p\coloneqq\sqrt{\frac{f}{\rho}} $ l\"ost $ (R) $; l\"ose $ (R) $ unter $ v(0)=0 $.\\
	L\"ose dazu \[ w'=-2\rho\sqrt{\frac{f}{\rho}}w-\rho w^2=-2\sqrt{\rho f}w-\rho w^2 \]
	$ u\coloneqq\frac{1}{w} $ l\"ost $ u'=\frac{-w'}{w^2}=2\sqrt{\rho f}u+\rho $. Es folgt:
	\begin{align*}
	u(t)&=e^{2\sqrt{\rho f}t}\left(\int_0^t\rho e^{-2\sqrt{\rho f}s}\dd s+u(0)\right)\\
	&=e^{2\sqrt{\rho f}t}\left(-\frac{\rho}{2\sqrt{\rho f}}\left(e^{-2\sqrt{\rho f}t}-1\right)+u(0)\right)\\
	&=-\frac{1}{2}\sqrt{\frac{\rho}{f}}+\frac{1}{2}\sqrt{\frac{\rho}{f}}e^{2\sqrt{\rho f}t}+u(0)e^{2\sqrt{\rho f}t}
	\end{align*}
	\[ w(t)=\frac{1}{u(t)}=\frac{1}{-\frac{1}{2}\sqrt{\frac{\rho}{f}}e^{2\sqrt{\rho f}t}\left(u(0)+\frac{1}{2}\sqrt{\frac{\rho}{f}}\right)} \]
	\[ v(t)=\sqrt{\frac{f}{\rho}}+w(t)=\sqrt{\frac{f}{\rho}}-\frac{1}{\frac{1}{2}\sqrt{\frac{\rho}{f}}\left(1-e^{-2\sqrt{\rho f}t}\right)+u(0)e^{2\sqrt{\rho f}t}}; \]
	mit $ u(0)=\sqrt{\frac{f}{\rho}} $ wird $ v(0)=0 $.
	\[ v(t)=\sqrt{\frac{f}{\rho}}\tanh(\sqrt{f\rho}t)\leq\sqrt{\frac{f}{\rho}}\qquad\tanh=\frac{e^x-e^{-x}}{e^x+e^{-x}} \]
\end{beispiel*}
\begin{beispiel*}
	\[ y'=e^{-x}y^2+y-e^x \]
	hat die Form $ (R) $ mit $ a=1 $, $ b=e^{-x} $ und $ f=-e^x $; $ y_p(x)\coloneqq e^x $. $ w=y-e^x $ l\"ost
	\[ w'=(1+2e^{-x}e^x)w+e^{-x}w^2=3w+e^{-x}w^2. \]
	$ v=\frac{1}{w} $ l\"ost $ v'=-3ve^{-x} $.
	\[ v(x)=\left(v(0)-\frac{1}{2}\left(e^{2x}-1\right)\right)e^{-3x} \]
	\[ w(x)=\frac{w(0)e^{3x}}{1-\frac{w(0)}{2}\left(e^{2x}-1\right)} \]
	\[ y(x)=\frac{Ce^{3x}}{1-\frac{1}{2}\left(e^{2x}-1\right)C}-e^x,\quad C\neq 0 \]
\end{beispiel*}
\begin{lemma}
	Angenommen, $ a,b\in C^1(I) $, $ b $ habe keine Nullstelle; \[ F\coloneqq bf+\frac{1}{2}\left(a+\frac{b'}{b}\right)'-\left(\frac{1}{2}\left(a+\frac{b'}{b}\right)\right)^2. \]
	Dann gilt:
	\begin{enumerate}
		\item Ist $ z $ L\"osung zu \[ (R')\qquad z'=z^2+F \]
		so ist \[ y\coloneqq\frac{1}{b}\left(z-\frac{1}{2}\left(a+\frac{b'}{b}\right)\right) \]
		L\"osung zu $ (R) $.
		\item L\"ost $ w $ die (lineare DGL) $ w''+Fw=0 $ und hat keine Nullstelle, so ist $ z\coloneqq -\frac{w'}{w} $ eine L\"osung zu $ (R') $.
	\end{enumerate}
\end{lemma}
\begin{beweis}
	\begin{enumerate}
		\item Es gilt \begin{align*} z'-z^2-F&=b'y+by'+\frac{1}{2}\left(a+\frac{b'}{b}\right)'-b^2y^2-b\left(a+\frac{b'}{b}\right)y-\left(\frac{1}{2}\left(a+\frac{b}{b'}\right)\right)^2-bf-\frac{1}{2}\left(a+\frac{b'}{b}\right)'+\left(\frac{1}{2}\left(a+\frac{b'}{b}\right)\right)^2\\
		&=b'y+by'-b^2y^2-aby-b'y-bf\\
		&=b(y'-(ay+by^2+f)). \end{align*}
		Also $ z $ l\"ost $ (R')\Leftrightarrow y $ l\"ost $ (R) $.
		\item $ z=-\frac{w'}{w} $
		\[ z'-z^2-F=-\frac{w''}{w}+\frac{w'^2}{w^2}-\left(\frac{w'}{w}\right)^2-F=-\frac{1}{w}(w''+Fw). \]
		Hieraus folgt schon die Behauptung.
	\end{enumerate}
\end{beweis}
\begin{beispiel*}
	Was sind die L\"osungen zu \[ y'=-\left(2+\frac{1}{t}\right)y+\frac{1}{t}y^2+t+2? \]
	Das ist $ (R) $ mit $ a=-2-\frac{1}{t} $, $ b=\frac{1}{t} $ und $ f=t+2 $ auf $ I\subset]0,\infty[ $. Berechne $ F: $
	\begin{align*} F&=bf+\frac{1}{2}\left(a+\frac{b'}{b}\right)'-\frac{1}{4}\left(a+\frac{b'}{b}\right)^2\\
	&=1+\frac{2}{t}+\frac{1}{t^2}-\left(1+\frac{1}{t}\right)^2\\
	&=0
	\end{align*}
	Also lautet $ (R') $ $ z'=z^2 $, also $ -\frac{z'}{z^2}=-1 $, also $ \left(\frac{1}{z}\right)'=-1 $; w\"ahle $ z(t)=-\frac{1}{t} $, also L\"osung zu $ (R') $.
	\[ y_p(t)\coloneqq\frac{1}{b}\left(z-\frac{1}{2}\left(a+\frac{b'}{b}\right)\right)=t\left(-\frac{1}{t}+1+\frac{1}{t}\right)=t \]
	l\"ost $ (R) $.\\
	Zu den weiteren L\"osungen: L\"ose
	\begin{align*} (B)\qquad v'=(a+2by_p)v+bv^2 \end{align*}
	d.h.
	\[ v'=\left(-2-\frac{1}{t}+2\frac{1}{t}t\right)v+\frac{1}{t}v^2=-\frac{1}{t}(v-v^2) \]
	also
	\[ \frac{v'}{v^2-v}=-\frac{1}{t}, \]
	nun ist
	\[ \frac{1}{v^2-v}=-\left(\frac{1}{v}-\frac{1}{v-1}\right)\Rightarrow \frac{v'}{v^2-v}=\left(\log\frac{v-1}{v}\right)'\Rightarrow\log\frac{v-1}{v}=-\log t+C_1\Rightarrow\frac{v-1}{v}=\frac{C}{t}, C=e^{C_1}, v(t)=\frac{C}{C-t} \]
	$ v $ ist auf $ ]0,C[ $ definiert, ebenso jede L\"osung $ y(t)=t+\frac{C}{C-t} $ zu $ (R) $.
\end{beispiel*}
\subsection{Differentialgleichungen mit trennbaren Variablen}
Sei $ I\subset\R $ ein Intervall, $ h\colon\hat I\rightarrow]0,\infty[ $ stetig.
\begin{satz}
	Sei $ g\in C^0(I) $, $ I'\subset I $, $ a\in I' $. Ist dann $ u\in C^1(I') $ eine L\"osung zur DGL
	\[ (S)\qquad y'=g\cdot h(y) \]
	(also $ u'(t)=g(t)h(u(t))\forall t\in I' $), so gilt
	\[ u=H_\ast^{-1}(H_\ast(u(a))+G) \]
	wobei $ H_\ast $ Stammfunktion zu $ \frac{1}{h} $ und $ G $ Stammfunktion zu $ g $ mit $ G(a)=0 $ ist.
\end{satz}
\begin{beweis}
	\[ \frac{u'}{h\circ u}=g=G'\Rightarrow (H_\ast\circ u)'=G' \]
	\[ H_\ast(u(t))-H_\ast(u(a))=\int_a^t(H_\ast\circ u(s))'\dd s=G(t),\quad H_\ast(x)=\int_c^x\frac{\dd z}{h(z)}, \]
	wenn $ \hat I=[c,d] $.
	\[ H_\ast(u(t))=H_\ast(u(a))+G(t) \] nach $ u(t) $ aufl\"osen. Hieraus folgt dann die Behauptung.
\end{beweis}
\begin{beispiel*}
	L\"osung zu \[ u'\sqrt{1-t^2}+\sqrt{1-u^2}=0,\quad u(0)=\frac{1}{2}. \]
	\begin{align*} &u'=-\frac{\sqrt{1-u^2}}{\sqrt{1-t^2}}\\\Rightarrow& \frac{u'}{\sqrt{1-u^2}}=-\frac{1}{\sqrt{1-t^2}}\\\Rightarrow& (\arcsin u)'(t)=-(\arcsin t)'\\\Rightarrow&\arcsin u(t)-\arcsin\frac{1}{2}=-\arcsin t\\\Rightarrow&\arcsin u(t)=\frac{\pi}{6}-\arcsin t\\\Rightarrow& u(t)=-\sin\left(\arcsin t-\frac{\pi}{6}\right)=-t\cos\frac{\pi}{6}-\frac{1}{2}\cos(\arcsin t)=\frac{-\sqrt{3}}{2}t-\frac{1}{2}\sqrt{1-t^2} \end{align*}
\end{beispiel*}
\subsection{Differentialgleichungen vom Typ $ y'=f(at+by+c) $, $ a,b,c\in\R $, $ f\in C^0(\R) $}
$ v(t)=by(t)+at+c $ muss $ v'(t)=by'+a=bf(v(t))+a $ ($ (S) $ mit $ g\equiv 1 $) l\"osen.
\begin{beispiel*}
	\[ y'=(2t+3y+1)^2,\quad f(z)=z^2, a=2, b=3, c=1. \]
	$ v(t)=3y(t)+2t+1 $ l\"ost \begin{align*} &v'(t)=3v(t)^2+2=3\left(v^2+\frac{2}{3}\right)\\\Rightarrow&\frac{v'}{v^2+\frac{2}{3}}=3\\\Rightarrow&\frac{v'}{\left(\sqrt{\frac{3}{2}}v\right)^2+1}=2\\\Rightarrow&\frac{\sqrt{\frac{3}{2}}v'}{1+\left(\sqrt{\frac{3}{2}}v\right)^2}=\sqrt{6}\\\Rightarrow&\left(\arctan\left(\sqrt{\frac{e}{2}}v\right)\right)'=\sqrt{6}\\\Rightarrow&\arctan\left(\sqrt{\frac{3}{2}}v\right)=\sqrt{6}t+C,\quad C=\arctan\sqrt{\frac{3}{2}}v(0) \end{align*}
	\[ v(t)=\sqrt{\frac{2}{3}}\tan(\sqrt{6}t+C), \]
	wenn $ -\frac{\pi}{2}<\sqrt{6}t+C<\frac{\pi}{2} $.
	\[ y(t)=\frac{1}{3}(v(t)-2t-1) \]
	ist L\"osung.
\end{beispiel*}
\subsection{Differentialgleichungen vom Typ $ y'=f\left(\frac{y}{t}\right) $, $ f\in C^0(\R^+) $}
$ u(t)\coloneqq\frac{y(t)}{t} $ muss \begin{align*} u'&=\frac{y'}{t}-\frac{y}{t^2}\\&=\frac{1}{t}\left(y'-\frac{y}{t}\right)\\&=\frac{1}{t}\left(f(u(t))-u(t)\right)\\&=g(t)h(u(t)),\quad g(t)=\frac{1}{t},h(\tilde x)=f(\tilde x)-\tilde x \end{align*}
l\"osen.
\begin{beispiel*}[Scheinwerfer]
	\[ \left(y'+\frac{x}{y}\right)^2=1+\left(\frac{x}{y}\right)^2 \]
	\[ y'=-\frac{x}{y}+\sqrt{1+\left(\frac{x}{y}\right)^2}=\frac{1}{\frac{x}{y}+\sqrt{1+\left(\frac{x}{y}\right)^2}}=f\left(\frac{y}{x}\right), \]
	mit $ f(s)=\frac{s}{1+\sqrt{1+s^2}} $; $ u(x)=\frac{y(x)}{x} $ l\"ost $ \frac{u'}{f(u)-u}=\frac{1}{x} $; $ U(s)\coloneqq\log\frac{
	s^2}{1+\sqrt{1+s^2}} $ l\"ost $ U'=\frac{1}{f(s)-s} $. Also
	\begin{align*}
	&(U\circ u)'=\frac{1}{x}=(\log x)'\\
	\Rightarrow&U(u(x))=\log x+C_1\\
	\Rightarrow&\frac{u(x)^2}{1+\sqrt{1+u(x)^2}}=Cx,\quad C=e^{C_1}\\
	\Rightarrow&\frac{y(x)^2}{x^2+x\sqrt{x^2+y(x)^2}}=Cx\\
	\Rightarrow&y(x)^2=2C_2x+C_2^2\quad(C_2>0\text{ passend})
	\end{align*}
\end{beispiel*}
\subsection{Differentialgleichungen vom Typ $ y'=f\left(\frac{a_1t+b_1y+c_1}{a_2t+b_yt+c_2}\right) $, $ a_k,b_k,c_k\in\R $, $ k=1,2 $}
\begin{description}
	\item[1. Fall:] $ \left|\begin{smallmatrix}
	a_1&b_1\\a_2&b_2
	\end{smallmatrix}\right|\neq 0 $; w\"ahle $ \binom{\zeta}{\eta}\in\R^2 $ mit 
	\[ \begin{pmatrix}
	a_1&b_1\\a_2&b_2
	\end{pmatrix} \begin{pmatrix}
	\zeta\\\eta
	\end{pmatrix}=-\begin{pmatrix}
	c_1\\c_2
	\end{pmatrix}. \]
	Dann ist $ w(s)=y(s+\zeta)-\eta $ L\"osung zu 
	\[ w'(s)=f\left(\frac{a_1s+a_1\zeta+b_1w+b_1\eta+c_1}{a_2s+a_2\zeta+b_2w+b_2\eta+c_2}\right)=f\left(\frac{a_1s+b_1w}{a_2s+b_2w}\right)=F\left(\frac{w(s)}{s}\right),\quad F(z)=f\left(\frac{a_1+b_1z}{a_2+b_2z}\right). \]
	$ w $ l\"ost eine DGL vom Typ 3.3.5.
	\item[2. Fall:] $ \left(\begin{smallmatrix}
	a_1&b_1\\a_2&b_2
	\end{smallmatrix}\right) $ ist nicht invertierbar. Wenn $ a_2=b_2=0 $, so $ y'=f\left(\frac{a_1}{c_1}t+\frac{b_1}{c_2}y+\frac{c_1}{c_2}\right) $ f\"allt unter Fall 3.3.4.\\
	Sei $ (a_2,b_2)\neq(0,0) $, dann $ (a_1,b_1)=\lambda(a_2,b_2) $, $ \lambda\in\R $. $ u\coloneqq a_2t+b_2y $ l\"ost \[ u'=a_2+b_2y'=a_2+b_2f\left(\frac{\lambda u+c_1}{u+c_2}\right). \]
	Keine Variable $ t $ tritt rechts auf. \\
	$ F $ sei Stammfunktion zu $ \frac{1}{a_2+b_2 f\left(\frac{\lambda x+c_1}{x+c_2}\right)} $. Dann $ \frac{u'}{a_2+b_2 f\left(\frac{\lambda u+c_1}{u+c_2}\right)}=1 $, $ (F\circ u)'=1 $, $ F(u(t))=t+C $.
\end{description}
\begin{beispiel*}
\begin{enumerate}
	\item[]
	\item \[ y'=-\frac{4t+3y-1}{3t+4y+1},\quad f(x)\coloneqq -x \]
	$ a_1=4 $, $ b_1=3 $, $ c_1=-1 $, $ b_2=4 $, $ c_2=1 $, $ \left|\begin{smallmatrix}
	4&3\\3&4
	\end{smallmatrix}\right|=7 $; $ \left(\begin{smallmatrix}
	4&3\\3&4
	\end{smallmatrix}\right)\left(\begin{smallmatrix}
	\zeta\\\eta
	\end{smallmatrix}\right)=\left(\begin{smallmatrix}
	1\\-1
	\end{smallmatrix}\right) $ l\"osen: $ \zeta=1,\eta=-1 $.\\
	$ w(s)=y(s+1)+1 $ l\"ost $ w'=-\frac{4s+3w}{3s+4w} $, $ v=\frac{w}{s} $ l\"ost
	\[ v'=-\frac{1}{s}\left(v+\frac{4+3v}{3+4v}\right)=-\frac{1}{s}\frac{4v^2+6v+4}{4v+3}\Rightarrow\frac{4v+3}{4v^2+6v+4}v'=-\frac{1}{s} \]
	\[ \log(4v^2+6v+4)'=(-2\log s)' \]
	\[ \log(4v^2(s)+6v(s)+4)=-2\log s+\log(4v(1)^2+6v(1)+4) \]
	nahe $ s=1 $.
	\[ 4v(s)^2+6v(s)+4=\underbrace{(4v(1)^2+6v(1)+4)}_{\eqqcolon C}\frac{1}{s^2} \]
	\[ v(s)^2+\frac{3}{2}v(s)=\frac{C}{4s^2}-1 \]
	\[ \left(v(s)+\frac{3}{4}\right)^2=\frac{C}{4s^2}-\frac{7}{16}=\frac{4c-7s^2}{16s^2},\quad |s|<\frac{4C}{7} \]
	\[ v(s)=-\frac{3}{4}+\frac{1}{4s}\sqrt{4C-7s^2} \]
	\[ w(s)=-\frac{3}{4}s+\sqrt{4C-7s^2} \]
	\[ y(t)=w(t-1)-1=-\frac{3}{4}(t-1)+\sqrt{4C-7(t-1)^2},\quad w(1)=v(1), y(2)=w(1)-1 \]
	\item \[ y'=\frac{4t+2y+3}{2t+y+1} \]
	$ \left|\begin{smallmatrix}
	4&2\\2&1
	\end{smallmatrix}\right|=0 $. $ u(t)=2t+y $ l\"ost
	\[ u'=2+y'=2+\frac{2u+3}{u+1}=\frac{4u+5}{u+1}\Rightarrow \frac{u+1}{4u+5}u'=1 \]
	\[ \left(\frac{u+\frac{5}{4}}{4u+5}-\frac{1}{4}\frac{1}{4u+5}\right)u'=1 \]
	\[ \left(1-\frac{1}{4u+5}\right)u'=4\Rightarrow \left(u-\frac{1}{4}\log(4u+5)\right)'=4 \]
	\[ u(t)=\frac{1}{4}\log(4u(t)+5)=4t+C \]
	\[ 2t+y(t)-\frac{1}{4}\log(8t+5+4y(t))=4t+C \]
	Geben wir vor: $ y(0)=0 $, so $ C=-\frac{1}{4}\log 5 $. Differenziere und werte in $ t=0 $ aus: Taylorpolynome f\"ur $ y(t) $ von beliebiger Ordnung berechenbar. $ y(0)=0, y'(0)=3 $; aus $ 2+y'(t)-\frac{y'(t)+8}{8t+5+4y(t)}=4 $, nochmaliges Ableiten: $ y''(0) $.
\end{enumerate}
\end{beispiel*}
\section{Systeme linearer Differentialgleichhungen erster Ordnung}
$ I\subset\R $ sei ein Intervall, $ t_0\in I $; $ A\colon I\rightarrow\K^{n\times n} $ sei stetig. Studiere $ u'=A\cdot u $ f\"ur $ u\colon I\rightarrow\K^n $ (dabei $ \K=\R $ oder $ \K=\C $).
\changesection
\begin{satz}
	Ist $ u_0\in\K^n $, so hat das AWP
	\[ \begin{cases}
	y'=Ay\\y(t_0)=u_0
	\end{cases} \]
	genau eine L\"osung $ u\colon I\rightarrow\K^n $.
\end{satz}
\begin{beweis}
	\begin{description}
		\item[Existenz:] \[ u_1(t)\coloneqq u_0+\int_{t_0}^t A(s)u_0\dd s,\quad t\in I \]
		Ist $ u_k\colon I\rightarrow\K^n $ schon definiert, so setze \[ u_{k+1}(t)=u_0+\int_{t_0}^t A(s)u_k(s)\dd s.\qquad (+) \]
		Behauptung: Auf jedem kompakten Intervall $ J\subset I $, $ t_0\in J $, konvergiert die Folge $ (u_k)_k $ gleichm\"a\ss ig.\\
		Dazu: F\"ur $ j>l $ ist
		\[ u_j(t)-u_l(t)=\sum_{k=l}^{j-1}(u_{k+1}(t)-u_k(t)). \]
		Also
		\[ |u_j(t)-u_l(t)|\leq\sum_{k=l}^{j-1}|u_{k+1}(t)-u_k(t)| \]
		$ M\coloneqq\sup_{x\in J}\norm{A(x)} $:
		\[ |u_{k+1}(t)-u_k(t)|\leq\frac{M^{k+1}}{(k+1)!}|u_0||t-t_0|^{k+1} \]
		Induktion nach $ k $: $ k=0 $:
		\[ |u_1(t)-u_0|=\left|\int_{t_0}^t A(s)u_0\dd s\right|\leq M|u_0||t-t_0| \]
		Gilt die Ungleichung f\"ur $ k-1 $, so auch f\"ur $ k $:
		\begin{align*}
		u_{k+1}(t)-u_k(t)&=\int_{t_0}^t A(s)u_k(s)\dd s-\int_{t_0}^t A(s)u_{k-1}(s)\dd s\\
		&=\int_{t_0}^t A(s)(u_k(s)-u_{k-1}(s))\dd s
		\end{align*}
		\begin{align*}
		|u_{k+1}(t)-u_k(t)|&\leq\int_{t_0}^t\norm{A(s)}|u_{k}(s)-u_{k-1}(s)|\dd s\\
		&\leq M\int_{t_0}^t\frac{M^k}{k!}|u_0|(s-t_0)^k\dd s\\
		&=\frac{M^{k+1}}{(k+1)!}|u_0||t-t_0|^{k+1}
		\end{align*}
		Ist $ J\subset [t_0-R,t_0+R] $, so \[ \max_{t\in J}|u_{k+1}(t)-u_k(t)|\leq\frac{(MR)^{k+1}}{(k+1)!}|u_0|. \]
		Also
		\[ \max_{t\in J}|u_j(t)-u_l(t)|\leq|u_0|\sum_{k=l}^{j-1}\frac{(MR)^{k+1}}{(k+1)!}. \]
		$ (u_j)_j $ erf\"ullt das Cauchykriterium in der gleichm\"a\ss igen Konvergenz auf $ J $.\\
		Sei $ u(t)\coloneqq\lim_{j\to\infty}u_j(t) $. Dann ist $ u $ stetig auf $ J $. Aus $ (+) $ folgt: $ u(t)=u_0\int_{t_0}^t A(s)u(s)\dd s $ ist stetig differenzierbar. Weiter $ u'(t)=A(t)u(t) $, $ u(t_0)=u_0 $, auf $ J $.
		\item[Eindeutigkeit auf $ J $:] Sei $ w\colon J\rightarrow\K^n $ eine L\"osung zu $ w'=Aw $ mit $ w(t_0)=0 $. Behauptung: $ w=0 $.\\
		Sei $ \delta\coloneqq\frac{1}{2M}>0 $. Wir zeigen $ w|_{[t_0,t_0+\delta]}=0 $, analog $ w|_{[t_0-\delta,t_0]}=0 $.\\
		F\"ur $ t_0\leq t\leq t_0+\delta $ gilt \[ w(t)=\int_{t_0}^t w'(s)\dd s=\int_{t_0}^t A(s)w(s)\dd s. \]
		Also
		\[ |w(t)|\leq\delta M\max_{[t_0,t_0+\delta]}|w|\Rightarrow \max_{[t_0,t_0+\delta]}|w|\leq\frac{1}{2}\max_{[t_0,t_0+\delta]}|w|. \]
		Hieraus folgt die Behauptung.
	\end{description}

\end{beweis}
Somit: $\begin{cases}
y'=Ay\\y(t_0)=u_0
\end{cases}$ hat auf jedem kompakten Intervall $ J\subset I $, $ t_0\in J $, genau eine L\"osung $ u_J $. W\"ahle Intervallfolge $ (J_k)_k $, $ t_0\in J_k $, $ J_k\subset J_{k+1} $, $ I=\bigcup_{k=1}^\infty J_k $. $ u_{J_{k+1}} $ und $ u_{J_k} $ l\"osen auf $ J_k $ dasselbe AWP. Hieraus folgt $ u_{J_{k+1}}|_{J_k}=u_{J_k} $. Setze also $ u(t)=u_{J_k}(t) $, wenn $ t\in J_k $.
%
%
%
%
%
\begin{beispiel*}
	$ A=\begin{pmatrix}
	-5&-6\\3&4
	\end{pmatrix} $, $ B(t)=\binom{e^t}{2} $, $ u_0=\binom{3}{1} $, $ t_0=0 $.
	\begin{description}
		\item[1. Schritt:] $ W $ bestimmen: \[ \chi_A(x)=\left|\begin{smallmatrix}x+5&6\\-3&x-4\end{smallmatrix}\right|=x^2+x-2=(x-1)(x+2). \]
		\[ A\binom{x_1}{x_2}=\binom{x_1}{x_2};\begin{smallmatrix}
		-5x_1-6x_2=-x_1\\x_1=-x_2
		\end{smallmatrix}\Rightarrow A\binom{1}{-1}=\begin{pmatrix}
		-5&-6\\3&4
		\end{pmatrix}\binom{1}{-1}=\binom{1}{-1} \]
		\[ A\binom{x_1}{x_2}=-2\binom{x_1}{x_2};\begin{smallmatrix}
		-5x_1-6x_2=-2x_1\\x_1=-2x_2
		\end{smallmatrix}\Rightarrow A\binom{-2}{1}=\begin{pmatrix}
		-5&-6\\3&4
		\end{pmatrix}\binom{-2}{1}=\binom{4}{-2} \]
		\[ W(t)=\begin{pmatrix}
		e^t&-2e^{-2t}\\
		-e^t&e^{-2t}
		\end{pmatrix},\quad W'= \begin{pmatrix}
		e^t&4e^{-2t}\\-e^t&-2e^{-2t}
		\end{pmatrix},\quad AW=\begin{pmatrix}
		-5&-6\\3&4
		\end{pmatrix}\begin{pmatrix}
		e^t&-2e^{-2t}\\-e^t&e^{-2t}
		\end{pmatrix}=\begin{pmatrix}
		e^t&4e^{-2t}\\-e^t&-2e^{-2t}
		\end{pmatrix}=W' \]
		\item[2. Schritt:] \[ W(s)^{-1}=\frac{1}{-e^{-s}}\begin{pmatrix}
		e^{-2s}&2e^{-2s}\\e^s&e^s
		\end{pmatrix}=-\begin{pmatrix}
		e^{-s}&2e^{-s}\\e^{2s}&e^{2s}
		\end{pmatrix} \]
		\[ W(0)^{-1}=-\begin{pmatrix}
		1&2\\1&1
		\end{pmatrix},\quad W(0)^{-1}\binom{3}{1}=-\binom{5}{4} \]
		\[ W(s)^{-1}B(s)=-\begin{pmatrix}
		e^{-s}&2e^{-s}\\e^{2s}&e^{2s}
		\end{pmatrix}\begin{pmatrix}
		e^s\\2
		\end{pmatrix}=\begin{pmatrix}
		1+4e^{-s}\\e^{3s}+2e^{2s}
		\end{pmatrix} \]
		\[ \int_0^t W(s)^{-1}B(s)\dd s=\int_0^t \begin{pmatrix}
		1+4e^{-s}\\e^{3s}+2e^{2s}
		\end{pmatrix}\dd s=\begin{pmatrix}
		t-4e^t+4\\\frac{1}{3}(e^{3t}-1)+e^{2t}-1
		\end{pmatrix}
			 \]
		Also haben wir:
		\[ u(t)=\begin{pmatrix}
		e^t&-2e^{-2t}\\-e^t&e^{-2t}
		\end{pmatrix}\left(\begin{pmatrix}
		-5\\-4
		\end{pmatrix}-\begin{pmatrix}
		t-4e^{-t}+4\\\frac{1}{3}e^{3t}+e^{2t}-\frac{4}{3}
		\end{pmatrix}\right) \]
	\end{description}
\end{beispiel*}
\changesubsection
\subsection{Exponentialfunktion einer Matrix}
Sei $ A\in\K^{n\times n} $; ist $ e^A $ wohldefiniert? Versuche
\[ e^A\coloneqq\sum_{m=0}^{\infty}\frac{A^m}{m!}. \]
Auf $ \K^{n\times n} $ hat man die Norm $ \norm{B}\coloneqq\sqrt{\sum_{k,l=1}^{n}|b_{kl}|^2} $. Es gilt $ \norm{B_1B_2}\leq\norm{B_1}\norm{B_2} $. Hieraus folgt $ \norm{A^m}\leq\norm{A}^m $. $ \left(\sum_{m=0}^{p}\frac{A^m}{m!}\right)_p $ ist also eine Cauchyfolge und konvergiert somit.
\begin{lemma}
	\bullshit
	\begin{enumerate}
		\item $ A,B\in\K^{n\times n} $, $ AB=BA $. Dann
		\[ e^{A+B}=e^Ae^B=e^Be^A. \]
		Insbesondere ist $ e^{-A}e^A=E_n $ ($ n\times n- $Einheitsmatrix).
		\item Ist $ A\in\K^{n\times n} $, $ S\in  GL(n,\K) $, so $ S^{-1}e^A S=e^{S^{-1}AS} $.
		\item Ist $ D=\begin{pmatrix}
		\lambda_1&\hdotsfor{1}&0\\\vdots&\ddots&\vdots\\
		0&\hdotsfor{1}&\lambda_n
		\end{pmatrix} $ mit $ \lambda_1,...,\lambda_n\in\K $, so
		\[ e^D=\begin{pmatrix}
		e^{\lambda_1}&\hdotsfor{1}&0\\\vdots&\ddots&\vdots\\0&\hdotsfor{1}&e^{\lambda_n}
		\end{pmatrix}. \]
		\item $ A\in\K^{n\times n} $, so ist $ W(t)=e^{At} $ Wronskimatrix zu $ y'=Ay $.
	\end{enumerate}
\end{lemma}
\begin{beweis}
	\begin{enumerate}
		\item \[ \frac{(A+B)^m}{m!}=\frac{1}{m!}\sum_{\mu=0}^{m}\binom{m}{\mu}A^\mu B^{m-\mu}=\sum_{\mu=0}^{m}\frac{A^\mu}{\mu!}\frac{B^{m-\mu}}{(m-\mu)!} \]
		Summiere \"uber alle $ m\in\N_0 $ und benutze Cauchy-Produkt-Regel.
		\item $ S^{-1}\frac{A^m}{m!}S=\frac{(S^{-1}AS)^m}{m!} $
		induktiv nach $ m $. Summiere \"uber alle $ m\in\N_0 $.
		\item Betrachte $ \frac{D^m}{m!} $. Die Behauptung folgt direkt.
		\item \[ e^{At}=\sum_{m=0}^{\infty}\frac{A^m}{m!}t^m,\quad \left(e^{At}\right)'=\sum_{m=1}^{\infty}\frac{A^m}{(m-1)!}t^{m-1}=A\sum_{m=1}^{\infty}\frac{(At)^{m-1}}{(m-1)!}=Ae^{At} \]
	\end{enumerate}
\end{beweis}
Sei $ A $ diagonalisierbar, also $ \exists S\in GL(n,\K) $ mit \[ S^{-1}AS=D=\begin{pmatrix}
\lambda_1\\&\ddots\\&&\lambda_n
\end{pmatrix},\quad S^{-1}e^A S=e^D. \]
Also $ e^A=Se^DS^{-1} $.
\begin{beispiel*}
	$ A=\begin{pmatrix}
	-5&-6\\3&4
	\end{pmatrix} $, $ S=\begin{pmatrix}
	1&2\\-1&-1
	\end{pmatrix} $, $ S^{-1}=\begin{pmatrix}
	-1-2\\1&1
	\end{pmatrix} $, $ D=\begin{pmatrix}
	1&0\\0&-2
	\end{pmatrix} $, $ e^D=\begin{pmatrix}
	e&0\\0&e^{-2}
	\end{pmatrix} $. Also:
	\[ e^A=\begin{pmatrix}
	1&2\\-1&-1
	\end{pmatrix}\begin{pmatrix}
	e&0\\0&e^{-2}
	\end{pmatrix}\begin{pmatrix}
	-1&-2\\1&1
	\end{pmatrix}=\begin{pmatrix}
	e&2e^{-2}\\-e&-e^{-2}
	\end{pmatrix}\begin{pmatrix}
	-1&-2\\1&1
	\end{pmatrix}=\begin{pmatrix}
	-e+2e^{-2}&-2e+2e^{-2}\\-e-e^{-2}&2e-e^{-2}
	\end{pmatrix} \]
	\[ e^{At}=S \begin{pmatrix}
	e^t&0\\0&e^{-2t}
	\end{pmatrix}S^{-1}=\begin{pmatrix}
	-e^t+2e^{-2t}&-2e^t+2e^{-2t}\\-e^t-e^{-2t}&2e^t-e^{-2t}
	\end{pmatrix} \]
\end{beispiel*}
Ist $ A $ allgemein und $ \K=\C $, so w\"ahle $ S\in GL(n,\C) $ mit \[ S^{-1}AS=A^J=\begin{pmatrix}
A_1\\&\ddots\\&&A_r
\end{pmatrix},\quad A_l=\begin{pmatrix}
\lambda_l&n_{l,1}\\&\ddots&n_{l,k_l}\\&&\lambda_l
\end{pmatrix}\in\C^{k_l\times k_l}, n_{l,1},...,n_{l,k_l}\in\lbrace 0,1\rbrace \]
$ A_l=\lambda_l E_{k_l}+N_l $, $ N_l $ nilpotent: $ N^{k_l}_l=0 $.
\[ e^{At}=Se^{A^Jt}S^{-1},\quad e^{A^Jt}=\begin{pmatrix}
a^{A_1t}\\&\ddots\\&&e^{A_rt}
\end{pmatrix} \]
\[ e^{A_lt}=e^{\lambda_l E_{k_l}+N_lt}=e^{\lambda_l E_{k_l}t}e^{N_lt}=e^{\lambda_l t}\sum_{\lambda=0}^{k_l-1}\frac{N_l^\lambda}{\lambda!}t^\lambda \]
\begin{beispiel*}
	\[ A=\begin{pmatrix}
	0&0&0&-8\\
	1&0&0&16\\
	0&1&0&-14\\
	0&0&1&6
	\end{pmatrix},\quad\chi_A(x)=|xE_4-A|=x^4-6x^3+14x^2-16x+8=(x-2)^2(x^2-2x+2) \]
    \begin{align*}
    &\eig(A,s)=\C \begin{pmatrix}
    -4\\6\\-4\\1
    \end{pmatrix}\\
    &\eig(A,1+i)=\C \begin{pmatrix}
    -4+4i\\8-4i\\-5+i\\1
    \end{pmatrix}\\
    &\eig(A,1-i)=\C \begin{pmatrix}
    -4-4i\\8+4i\\-5-i\\1
    \end{pmatrix}
    \end{align*}
    \[ (A-2e_4)^2=\begin{pmatrix}
    4&0&-8&-16\\
    -4&4&16&24\\
    1&-4&-10&-12\\
    0&1&2&2
    \end{pmatrix},\quad v_4\coloneqq \begin{pmatrix}
    2\\-2\\1\\0
    \end{pmatrix}\in\sH(A,2) \]
    \[ v_3=(A-2E_4)v_4=Av_4-2v_4,\quad Av_4=v_3+2v_4 \]
    $ S\coloneqq(v_1,\bar v_1,v_3,v_4) $ erf\"ullt
    \[ S^{-1}AS=A^J\coloneqq \begin{pmatrix}
    1+i&0&0&0\\
    0&1-i&0&0\\
    0&0&2&1\\
    0&0&0&2
    \end{pmatrix}. \]
    \[ e^{At}=S \begin{pmatrix}
    e^{(1+i)t}&0&0&0\\
    0&e^{(1-i)t}&0&0\\
    0&0&e^{2t}&te^{2t}\\
    0&0&0&e^{2t}
    \end{pmatrix}S^{-1} \]
\end{beispiel*}